% TeX root = ../../paper.tex

\begin{abstract}
	
	In modeling and design, both textual and visual artifacts have long been used~\cite{schindler2012werkzeuginfrastruktur}. Petri nets are a powerful tool used for asynchronous or non-deterministic processes in software development, industrial engineering~\cite{yakovlev1996petri} and other domains~\cite{chaouiya2007petri}. With powerful analysis toolkits they can be used not only for visualization but also to support modeling and verification~\cite{barros2007use}.
	
	We propose a textual modeling language, \emph{petrinets4analysis}, based on the MontiCore Language Workbench~\cite{rumpe2017monticore}, which affords the use of petri nets as a primary modeling artifact. Our domain specific language (DSL) is usable by programmers and non-programmers alike, and the corresponding analysis toolkit allows access to transformations, model checking, and printing functionality. Due to the powerful language generation by MontiCore based on a simple context-free grammar, only a small hand-written codebase is needed, enhancing maintainability and extensibility.
	
	Our work demonstrates the use of language workbenches to create a modeling DSL and the accompanying benefits, and can easily be expanded upon to include more advanced analysis techniques or transformations.
	 
	% 1.) Introduction. In one sentence, what’s the topic? Phrase it in a
	% way that your reader will understand. If you’re writing a PhD thesis,
	% your readers are the examiners – assume they are familiar with the
	% general field of research, so you need to tell them specifically what
	% topic your thesis addresses. Same advice works for scientific papers –
	% the readers are the peer reviewers, and eventually others in your
	% field interested in your research, so again they know the background
	% work, but want to know specifically what topic your paper covers.
	%
	
	%
	% 2.) State the problem you tackle. What’s the key research question? 
	% Again, in one sentence. (Note: For a more general essay, I’d adjust
	% this slightly to state the central question that you want to address)
	% Remember, your first sentence introduced the overall topic, so now you
	% can build on that, and focus on one key question within that topic. If
	% you can’t summarize your thesis/paper/essay in one key question, then
	% you don’t yet understand what you’re trying to write about. Keep
	% working at this step until you have a single, concise (and
	% understandable) question.
	
	%
	% 3.) Summarize (in one sentence) why nobody else has adequately
	% answered the research question yet. For a PhD thesis, you’ll have an
	% entire chapter, covering what’s been done previously in the
	% literature. Here you have to boil that down to one sentence. But
	% remember, the trick is not to try and cover all the various ways in
	% which people have tried and failed; the trick is to explain that
	% there’s this one particular approach that nobody else tried yet (hint:
	% it’s the thing that your research does). But here you’re phrasing it
	% in such a way that it’s clear it’s a gap in the literature. So use a
	% phrase such as “previous work has failed to address…”. (if you’re
	% writing a more general essay, you still need to summarize the source
	% material you’re drawing on, so you can pull the same trick – explain
	% in a few words what the general message in the source material is, but
	% expressed in terms of what’s missing)
	% 
	
	%  4.) Explain, in one sentence, how you tackled the research question.
	% What’s your big new idea? (Again for a more general essay, you might
	% want to adapt this slightly: what’s the new perspective you have
	% adopted? or: What’s your overall view on the question you introduced
	% in step 2?)
	
	%  5.) In one sentence, how did you go about doing the research that
	% follows from your big idea. Did you run experiments? Build a piece of
	% software? Carry out case studies? This is likely to be the longest
	% sentence, especially if it’s a PhD thesis – after all you’re probably
	% covering several years worth of research. But don’t overdo it – we’re
	% still looking for a sentence that you could read aloud without having
	% to stop for breath. Remember, the word ‘abstract’ means a summary of
	% the main ideas with most of the detail left out. So feel free to omit
	% detail! (For those of you who got this far and are still insisting on
	% writing an essay rather than signing up for a PhD, this sentence is
	% really an elaboration of sentence 4 – explore the consequences of your
	% new perspective).
	
	%  6.) As a single sentence, what’s the key impact of your research?
	% Here we’re not looking for the outcome of an experiment. We’re looking
	% for a summary of the implications. What’s it all mean? Why should
	% other people care? What can they do with your research. (Essay folks:
	% all the same questions apply: what conclusions did you draw, and why
	% would anyone care about them?)
\end{abstract}
