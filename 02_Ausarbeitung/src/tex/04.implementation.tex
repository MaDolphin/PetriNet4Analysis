% TeX root = ../../paper.tex

\section{Implementation}\label{sec:implementation}

The \emph{petrinets4analysis} language and toolkit was developed using the MontiCore Language Workbench~\cite{rumpe2017monticore}, a parser generator, development framework, and runtime environment specialized for the design of DSLs. We use the grammar in Listing~\ref{lst:language:grammar} as input for MontiCore, which creates in turn~\cite{rumpe2017monticore}
\begin{itemize}
	\item the parser for our \emph{petrinets4analysis} DSL (by invoking the ANTLR parser generator~\cite{parr1995antlr}), including symbol resolution for place and transition names,
	\item classes for the AST's non-terminals, given in Figure~\ref{fig:language:ast-uml},
	\item development infrastructure, such as code for visitors and generation.
\end{itemize}

\begin{figure}[htb]
	\centering
	\makebox[\textwidth][c]{
		\begin{tikzpicture}
		\umlclass[x=0,y=0.5]{ASTPetrinet}{
			+ name : String
		}{}
		\umlsimpleclass[x=-2,y=-0.7,anchor=north]{ASTAssertion}
		\umlHVaggreg[mult2=*,pos2=1.9]{ASTPetrinet}{ASTAssertion}
		\umlclass[type=interface,x=-4,y=-2.8]{ASTRequirement}{}{
			+ \emph{verify}(ASTPetrinet) : Optional<Boolean>
		}
		\umlHVaggreg[mult2=1,pos2=1.8]{ASTAssertion}{ASTRequirement}
		\umlsimpleclass[x=-6,y=-4.5]{ASTGlobalRequirement}
		\umlHVimpl[anchor2=-50]{ASTGlobalRequirement}{ASTRequirement}
		\umlsimpleclass[x=-6,y=-5.5]{ASTSubclassRequirement}
		\umlHVimpl[anchor2=-50]{ASTSubclassRequirement}{ASTRequirement}
		\umlsimpleclass[x=-1,y=-5]{ASTLiveness}
		\umlHVimpl[anchor2=-50]{ASTLiveness}{ASTRequirement}
		\umlclass[x=4.7,y=-0.7,anchor=north]{ASTPlace}{
			+ name : String
		}{}
		\umlHVaggreg[mult2=*,pos2=1.9]{ASTPetrinet}{ASTPlace}
		\umlclass[x=1.7,y=-0.7,anchor=north]{ASTTransition}{
			+ name : String
		}{}
		\umlHVaggreg[mult2=*,pos2=1.9]{ASTPetrinet}{ASTTransition}
		\umlsimpleclass[x=0.8,y=-3,anchor=west]{ASTFromEdge}
		\umlVHaggreg[pos2=1.8,anchor1=-150,attr2=|*]{ASTTransition}{ASTFromEdge}
		\umlsimpleclass[x=0.8,y=-4,anchor=west]{ASTToEdge}
		\umlVHaggreg[pos2=1.8,anchor1=-150,attr2=|*]{ASTTransition}{ASTToEdge}
		\umlVHuniassoc[attr2=|*,anchor1=-100,anchor2=10,pos2=1.9]{ASTPlace}{ASTFromEdge}
		\umlVHuniassoc[attr2=|*,anchor1=-100,anchor2=10,pos2=1.9]{ASTPlace}{ASTToEdge}
		\umlsimpleclass[type=abstract,x=4,y=-4.8]{ASTEdge}
		\umlHVuniassoc[attr2=|1,pos2=1.9,anchor2=-60]{ASTEdge}{ASTPlace}
		\umlHVinherit[anchor1=-10]{ASTFromEdge}{ASTEdge}
		\umlHVinherit[anchor1=-10]{ASTToEdge}{ASTEdge}
		\umlsimpleclass[x=3,y=-5.8]{ASTIntLiteral}
		\umlHVHaggreg[attr2=1|weight,align2=right,anchors=west and west,arm2=-1,pos2=2.9]{ASTEdge}{ASTIntLiteral}
		\umlVHaggreg[attr2=0..1|initial,align2=left,anchor1=-35,pos2=1.9]{ASTPlace}{ASTIntLiteral}
		\umlVHuniassoc[mult2=*,pos2=1.7,anchor1=25]{ASTLiveness}{ASTTransition}
		\end{tikzpicture}
	}
	\caption{Structure of our AST's non-terminals}\label{fig:language:ast-uml}
\end{figure}

Hence, only language-specific additions such as analysis methods have to be implemented, resulting in a small handwritten codebase. Besides the AST and coco classes discussed in Section~\ref{sec:language}, we group our implementation in three major packages; Figure~\ref{fig:impl:uml} shows their interrelations.

\begin{figure}[htb]
	\centering
	\makebox[\textwidth][c]{
		\begin{tikzpicture}
		\umlsimpleclass[x=0,y=0]{PetrinetsTool}
		\umlemptypackage[x=-0.2,y=2]{cocos}
		\umlVHVuniassoc{PetrinetsTool}{cocos}
		\umlemptypackage[x=1.7,y=2,name=ast]{\_ast}
		\umlVHVuniassoc[anchor1=30]{PetrinetsTool}{ast}
		\begin{umlpackage}{prettyprint}
		\umlsimpleclass[x=3,y=2,anchor=west]{PetrinetPrettyPrinter}
		\umlsimpleclass[x=3,y=1,anchor=west]{PetrinetDotPrinter}
		\end{umlpackage}
		\umlHVHuniassoc[anchor1=10,arm1=1.3]{PetrinetsTool}{PetrinetPrettyPrinter}
		\umlHVHuniassoc[anchor1=10,arm1=1.3]{PetrinetsTool}{PetrinetDotPrinter}
		\umlsimpleclass[x=7.7,y=3]{IndentPrinter}
		\umlHVinherit{PetrinetPrettyPrinter}{IndentPrinter}
		\umlHVinherit{PetrinetDotPrinter}{IndentPrinter}
		\begin{umlpackage}{transformations}
		\umlsimpleclass[x=-3.5,y=1]{Transformation}
		\umlsimpleclass[x=-3.5,y=2]{SimpleTransformations}
		\umlVHVuniassoc{Transformation}{SimpleTransformations}
		\end{umlpackage}
		\umlVHuniassoc[anchor1=160]{PetrinetsTool}{Transformation}
		\begin{umlpackage}{analyses}
		\umlsimpleclass[x=-3,y=-1]{Subclass}
		\umlsimpleclass[x=0,y=-1]{Liveness}
		\umlsimpleclass[x=3,y=-1]{Boundedness}
		\umlsimpleclass[x=1.3,y=-2.5]{CoverabilityTree}
		\umlHVuniassoc{Liveness}{CoverabilityTree}
		\umlHVuniassoc{Boundedness}{CoverabilityTree}
		\umlnote[x=-2,y=-2.3,width=1.5cm]{Liveness}{Checker}
		\umlnote[x=-2,y=-2.3,width=1.5cm]{Boundedness}{Checker}
		\umlnote[x=-2,y=-2.3,width=1.5cm]{Subclass}{Checker}
		\umlsimpleclass[x=4.3,y=-2.5]{Marking}
		\umluniaggreg[mult2=1]{CoverabilityTree}{Marking}
		\umlsimpleclass[x=7,y=-2.5]{TokenCount}
		\umluniaggreg[mult2=*]{Marking}{TokenCount}
		\umlsimpleclass[x=7,y=-1]{InfiniteTokenCount}
		\umlinherit{InfiniteTokenCount}{TokenCount}
		\umldep[anchor1=12,anchor2=-170]{CoverabilityTree}{InfiniteTokenCount}
		\umldep{Boundedness}{InfiniteTokenCount}
		\umldep[anchor2=165]{Boundedness}{TokenCount}
		\umlVHVdep[anchor1=-12,anchor2=-165,arm1=-0.15]{CoverabilityTree}{TokenCount}
		\end{umlpackage}
		\umlHVHuniassoc{PetrinetsTool}{Subclass}
		\umlVHVuniassoc{PetrinetsTool}{Liveness}
		\umlHVuniassoc[anchor1=-10]{PetrinetsTool}{Boundedness}
		\umldep{analyses}{ast}
		\umldep{cocos}{ast}
		\umlCNdep{transformations}{0.2,1.3}{ast}
		\umldep{prettyprint}{ast}
		\end{tikzpicture}
	}
	\caption{Class diagram for the analysis toolkit}\label{fig:impl:uml}
\end{figure}

\subsubsection{transformations}

This package implements the six simple transformations introduced by Murata~\cite{murata1989petri} and described in Section~\ref{sec:background:transformations}. The \texttt{Transformation} class will apply them to a petri net until no more changes are possible.

\subsubsection{analyses}

In the analyses package, the coverability tree method (Section~\ref{sec:background:covtree}) is implemented for verifying liveness, safeness and boundedness. Additionally, subclasses (Section~\ref{sec:background:properties}) are defined and evaluated here.

This package is intended as the main location for analysis code which can also be referred to from the AST classes.

\subsubsection{prettyprint}

We provide two printing classes both relying on MontiCore's built-in \texttt{IndentPrinter} visitor. The \texttt{PrettyPrinter} outputs a syntactically correct petri net which parses to the given AST representation and is formatted according to common guidelines (e.g.\ indentation and whitespace). 

The \texttt{DotPrinter} allows transformation of the model into the popular graph description language \emph{Dot}~\cite{gansner2000open} for subsequent processing with other compatible tools. In particular, \emph{graphviz}~\cite{gansner2000open} could be used to draw the net with appropriate positioning.

\subsection{Tool Functionality}

In addition to the three given packages, the \texttt{PetrinetsTool} is the entry point for a command-line analysis application. After a petri net has been parsed and verified, it provides an interface to access the transformation, analysis, and printing functionality.

For property verification, this mechanism is an alternative to the enforcement with assertions (cf.\ Section~\ref{sec:lang:semantics}), in the form of a query and result. Additionally, a short ``proof'' can be displayed, e.g. a transition sequence leading to violation of boundedness.

\subsection{Transformations}\label{sec:impl:speed}

The coverability tree introduced in Section~\ref{sec:background:covtree} provides an algorithm to determine liveness, boundedness and safeness. It is total, but not necessarily efficient. Hence, transformations are essential to speed up computation time.

As an example, Figure~\ref{fig:impl:cookie} describes a cookie vending machine switching between a \emph{Selection} and \emph{Output} mode. After inserting a coin to get to \emph{Selection}, the customer can press the button ``+'' to select a number of cookies that will be taken from a supply and dispensed to the tray in \emph{Output} mode.

The coverability tree computation for this petri net takes $6.42$s ($\pm0.49$s)\footnote{\label{fn:experiment}53 repetitions, first 3 discarded, 95\% confidence interval} on a local machine. This is because every ordering of \emph{Take Cookie}, \emph{Dispense} and steps in the \emph{Insert}--\emph{Accept Coin}--\emph{Change Mode} cycle need to be considered, leading to an exponential size in the number of tokens (``cookies'') in the supply.

We perform the implemented transformations to remove \emph{Take Cookie}, \emph{Power}, and \emph{Insert}. This reduces the potential transition sequences and leads to a computation time---including transformations---of $0.92$s ($\pm0.06$s)\textsuperscript{\ref{fn:experiment}} on the same machine.

\begin{figure}[htb]
	\centering
	\begin{tikzpicture}
		\node [place,label=below:Selection] (sm) {};
		\node [place,tokens=1,below of=sm,label=below:Output] (om) {};
		\node [place,right=2cm of sm,label=above:Count] (c) {};
		\node [transition,right of=sm,label=right:{Press ``+''},rotate=90] (bp) {}
			edge[pre,bend left=30] (sm) edge[post,bend right=30] (sm)
			edge[post] (c);
		\node [place,below of=c,label=below:Tray] (t) {};
		\node [place,tokens=4,below of=t,label=below:Supply] (s) {};
		\node [place,tokens=1,below right=0.85cm and 0.85cm of om,label=below:Power] (p) {};
		\node [transition,right of=om,label=below right:{\footnotemark[1]},rotate=90] (d) {}
			edge[pre,bend left=30] (om) edge[post,bend right=30] (om)
			edge[pre,bend left=15] (p) edge[post,bend right=15] (p)
			edge[pre] (s) edge[post] (t) edge[pre,bend right=20] (c);
		\node [transition,right of=t,label={[align=center]right:{Take\\Cookie}},rotate=90] (tk) {}
			edge[pre] (t);
		\node [place,left=2cm of sm,label=left:Cash] (cash) {};
		\node [place,below of=cash,label=left:Coin] (slot) {};
		\node [transition,left of=sm,rotate=90,label=right:{Accept Coin}] (acc) {}
			edge[pre,bend right=20] (om) edge[post] (sm) edge[pre] (slot) edge[post] (cash);
		\node [place,tokens=1,below left=0.85cm and 0.85cm of om,label=below:{Slot Open}] (so) {};
		\node [transition,left of=so,rotate=45,label=above left:Insert] (in) {}
			edge[pre] (so) edge[post] (slot);
		\node [transition,left of=om,rotate=135,label=above:{\footnotemark[2]}] (mode) {}
			edge[pre,bend left=30] (sm) edge[post] (om) edge[post] (so);
	\end{tikzpicture}
	\caption{A cookie vending machine, where transformations can provide a significant speed-up for analyses. Adapted from B. Koenig, \url{www.ti.inf.uni-due.de/fileadmin/public/teaching/mod/slides/ws201112/petri-2x2.pdf} \protect\footnotemark[1]Dispense \protect\footnotemark[2]Change Mode}\label{fig:impl:cookie}
\end{figure}

\subsection{Extensibility and Future Work}

Our application is designed to be easily extensible to new analyses or transformations. Additional assertions are added as a grammar non-terminal implementing the \texttt{Requirement} interface. Hence, they will have to override the \texttt{verify} method which returns the verification result and reference appropriate methods in the \texttt{analyses} package.

For instance, Murata describes a number of specialized theorems for properties such as higher liveness levels and reachability~\cite{murata1989petri}. Many of them only apply to petri nets with certain graph-theoretic properties, which could be realized by returning \texttt{Optional.empty()} if no theorem is applicable. Further we suggest implementing more efficient approaches to determine l1-liveness, safeness and boundedness, in particular for larger nets with relatively simple structure (e.g., Kemper and Bause~\cite{kemper1992efficient} describe a polynomial-time algorithm for free-choice nets).

In terms of transformations, edge weights---which complicate analyses and other transformations---could be eliminated through the addition of intermediate places. Additionally, more intricate and more powerful transformations can be added to the \texttt{transformations} package, hence further simplifying input petri nets. 
